\documentclass{article}
\usepackage{Sweave}
\usepackage[utf8]{inputenc}
\usepackage{fullpage}
\usepackage{url}
\usepackage{verbatim}
\usepackage{color}
\usepackage{hyperref}
\usepackage[usenames,dvipsnames,svgnames,table]{xcolor}
\usepackage{tikz}
\usetikzlibrary{shapes,arrows}
\definecolor{darkblue}{rgb}{0,0,0.5}

\begin{document}
\newcommand{\mcode} {\texttt{GS\_MotherCode.r}}
\newcommand{\bigtt}[1] {\textcolor{darkblue}{{\large {\texttt{#1}}}}}
%\hyphenation{GS\_-Mother-Code.r}

\noindent {\Large \bfseries Description: Code R pour l'analyse des
  données d'échantillonage}

\tableofcontents

% Define block styles
\tikzstyle{decision} = [diamond, draw,
    text width=4.5em, text badly centered, node distance=3cm, inner sep=0pt]
\tikzstyle{block} = [rectangle, draw,
    text width=5em, text centered, rounded corners, minimum height=4em]
    \tikzstyle{blockwide} = [rectangle, draw,
    text width=15em, text centered, rounded corners, minimum height=4em]
\tikzstyle{line} = [draw, -latex']
\tikzstyle{cloud} = [draw, ellipse,node distance=3cm,minimum height=2em]


\newpage

\begin{centering}
\begin{tikzpicture}[node distance = 2cm, auto]
    % Place nodes
    \node [block] (init) {Premier lancement du code?};
    \node [cloud, left of=init] (expert) {expert};
    \node [block, right of=init, node distance = 4cm] (1ere) {Voir section \ref{reglages1}};
    \node [block, below of=init, node distance = 3cm] (newdata) {Nouvelles données à importer?};
    \node [block, right of=newdata, node distance = 4cm] (newdatasec) {Voir section \ref{nouvdb}};

    \node [block, below of=newdata, node distance = 4cm] (filtre) {Filtre sur espèces ou
      filtre sur années à rajouter?};
    \node [decision, below of=filtre, node distance=4cm] (lancement)
    {Lancement des routines};
        \node [blockwide, left of=lancement, node distance=4cm] (manu)
    {Dans \mcode, spécifiez:\\ \texttt{sorties.INV = FALSE}\\
    \texttt{sorties.poissons = FALSE}\\ \texttt{sorties.INV = FALSE}};
        \node [blockwide, right of=lancement, node distance=4cm] (auto)
    {Dans \mcode, spécifiez:\\ \texttt{sorties.INV = TRUE}\\
    \texttt{sorties.poissons = TRUE}\\ \texttt{sorties.INV = TRUE}};
%    \node [block, below of=evaluate] (decide) {Dans mother\_code,
%      options FALSE, source(mother\_bcode), dans la console lancez
%      manuellement les fonctions désirées, voir section X};
%    \node [block, below of=decide, node distance=3cm] (stop) {Dans
%      mother\_code, options TRUE, source(mother\_code)}

    % Draw edges
    \path [line] (init) -- node {non} (newdata);
    \path [line] (newdata) -- node {non} (filtre);
%    \path [line] (evaluate) -- (decide);
%    \path [line] (decide) -| node [near start] {yes} (update);
%    \path [line] (filtre) -| (update);
%    \path [line] (decide) -- node {no}(stop);
    \path [line,dashed] (expert) -- (init);
    \path [line] (init) -- node {oui} (1ere) ;
    \path [line] (newdata) -- node {oui} (newdatasec) ;
    \path [line] (filtre) -- node {non} (lancement) ;
%    \path [line,dashed] (newdata) -- (1ere) ;

\end{tikzpicture}
\end{centering}
\newpage
\section{Accès rapide aux descriptions des fonctions R}
\subsection{Invertébrés}
Indices de biodiversité: Shannon, Piélou et Margalef, richesse
spécifique: \\

\noindent  \hyperlink{i1}{\bigtt{inv.biodiv(AS="pas de filtre",qunit="St",wC="all",save=FALSE)}} \\
  \hyperlink{i2}{\bigtt{inv.biodiv.geom(AS="A",save=FALSE)}}\\
  \hyperlink{i3}{\bigtt{inv.sprich.tbl(AS="A",grtax="Groupe",save=FALSE)}}\\
  \hyperlink{i4}{\bigtt{sprich.by.aggrtaxo(AS="A", grtax="Groupe", save=FALSE)}}\\
  \hyperlink{i5}{\bigtt{inv.RichSpecifique(AS="A", aj.impact=FALSE, aggr="St")}}\\

\noindent Densité moyenne sur divers facteurs (transect/station/géomorphologie),
avec l'option de conserver les 10 espèces les plus abondantes
seulement (tableaux et graphiques):\\
\hyperlink{i6}{\bigtt{inv.dens.tbl(AS="A", grtax="G\_Sp", smpl.unit="St",
      wZeroAll=FALSE, save=FALSE)}}\\
  \hyperlink{i7}{\bigtt{inv.dens.geom(AS="A", aj.impact=FALSE, spttcampagnes=FALSE, save=FALSE)}}\\
 \hyperlink{i8}{\bigtt{inv.graph.TS(AS="A", wtype="allsp",
      wff="Groupe", top10year="", save=TRUE)}}\\
 \hyperlink{i9}{\bigtt{inv.graph.TS.top10(AS="A",
          wff="Groupe", top10year="all", save=TRUE)}}\\

\subsection{Poissons}
\noindent Données brutes reformattées:\\
\hyperlink{p1} {\bigtt{poissons.tableau.brut(save=FALSE)}}\\
\hyperlink{p2} {\bigtt{BioDens.sp.poissons()}}\\

\noindent Densité, biomasse, richesse spécifique, etc. par catégorie d'espèce:\\
\hyperlink{p3} {\bigtt{poissons.ts1(AS="A",save=FALSE)}}\\

\noindent Densité moyenne (par station, année, toutes stations, toutes années):\\
\hyperlink{p4} {\bigtt{poissons.ts2(AS="A",save=FALSE)}}\\

\noindent Graphiques de séries temporelles biomasse, densité, etc.\\
\hyperlink{p5} {\bigtt{poissons.p3(quel.graph="all",save=FALSE)}}

\subsection{LIT}
\noindent Données brutes reformattées, couvertures moyennes, etc.:\\
\hyperlink{l1} {\bigtt{LIT.tableau.brut(save=FALSE, AS="pas de filtre")}}\\
\hyperlink{l2} {\bigtt{LIT.resume(yy=2011, ff="Coraux\_Gen", AS="A",
    save=FALSE)}}\\

\noindent Couverture moyenne par facteurs divers, différence de
couvertures entre zones impactées et non-impactées:\\
\hyperlink{l3} {\bigtt{LIT.ts1(AS="A")}}\\
\hyperlink{l4} {\bigtt{LIT.ts2(AS="A")}}\\

\noindent Graphiques en barres des couvertures moyennes par substrat:\\
\hyperlink{l5} {\bigtt{LIT.bp1(yy=2011, ff2="Coraux\_Gen", AS="A")}}

\clearpage
\section{Survol}

\subsection*{Réglages initiaux}
\label{reglages1}
Pour lancer le code et produire tous les tableaux/graphiques,
commencez par créer deux dossiers à l'emplacement de votre choix:
\begin{enumerate}
      \item un dossier où les analyses R sont faites
      [\texttt{dossier.R}], et qui contiendra les fichiers R requis par
      les analyses
  \item un dossier où les dernières versions des bases de données sont
    sauvegardées \texttt{[dossier.DB]} (ce dossier doit être
    synchronisé avec DropBox)
     \end{enumerate}

     Une fois ces dossiers créés, placez dans le fichier
     \texttt{dossier.R} tous les fichiers .r contenus dans le dossier
     \texttt{cw-ginger-soproner}, disponible en format ZIP à l'adresse
     suivante:
%     \url{https://github.com/lauratboyer/cw-ginger-soproner/archive/master.zip}. Ce
     site contient les dernières versions du code ainsi qu'une
     historique de toutes les modifications effectuées depuis Janvier 2013.

Ouvrez le fichier \texttt{GS\_MotherCode.r} \footnote{Un fichier .r
  peut être modifié sous NotePad ou l'éditeur
compris avec R. Si vous programmez plus souvent, \emph{TINN-R} et \emph{R-Studio}
sont deux alternatives disponibles gratuitement. Ces logiciels
permettent non seulement
d'éditer sous R, mais aussi d'interagir plus facilement avec la
console. } et définissez la valeur des objets \\ \texttt{dossier.R} et \texttt{dossier.DB}
en fonction de l'emplacement choisi pour ces dossiers.
Pour voir un exemple du format accepté par R pour la
  définition de l'emplacement de dossiers, tapez dans la console R:
\begin{Schunk}
\begin{Sinput}
> getwd() # montre le working directory
> print("hop")
\end{Sinput}
\end{Schunk}
  \noindent Le symbole \texttt{\#} dénote le début d'un commentaire non-exécuté
par R.\\


\noindent \emph{Note:} Lorsque non-existants, une fonction dans \texttt{GS\_MotherCode.r} crée
automatiquement les trois dossiers
suivants dans le \texttt{dossier.R}: \emph{Tableaux},
\emph{Graphiques} and \emph{Data}. \\


\subsection*{Incorporation de nouvelles versions des bases de données}
\label{nouvdb}
Les fichiers sont lus par R en format .csv car c'est plus rapide et
moins compliqué. Pour sauver les tableaux dans le format approprié,
ouvrir le tableau à sauver, aller dans ``Enregistrer sous... ``,
sélectionner dans le menu déroulant CSV (le premier CSV) -- attention
de sélectionner la première option et non CSV - MAC. Si l'ordinateur
est un MAC, utiliser l'option CSV - DOS.

   \subsection*{Routine de lancement des sorties de base}

   Dans le fichier \mcode, s'assurez que les variables sont bien définies dans
   la première section. Notamment, les variables
   \texttt{sorties.INV}, \texttt{sorties.LIT} et
   \texttt{sorties.POISSONS} devraient avoir la valeur \texttt{TRUE} si vous
   voulez produire les sorties associées, ou \texttt{FALSE} dans le
   cas opposé. La variable \texttt{filtre.sur.especes} contrôle les
   espèces incluses dans l'analyse: lorsque sa valeur est
   \texttt{FALSE} toutes les espèces sont incluses (voir prochaine
   section).

  Vérifiez aussi que l'objet \texttt{tourner.tout} (défini dans \mcode) a la valeur \texttt{TRUE} si vous voulez
tout tourner immédiatement, ou \texttt{FALSE} si vous voulez
charger les fonctions sous R et tourner certains
aspects de l'analyse manuellement.
\\
Ouvrez R \footnote{Une courte introduction aux concepts de bases de R peut
  se trouver dans les Sections 1 et 2 du document \emph{SimpleR}, disponible
  à l'adresse suivante:
  \url{cran.r-project.org/doc/contrib/Verzani-SimpleR.pdf} (voir aussi
  les autres sections qui survolent plusieurs applications
  statistiques).}, et chargez le fichier \mcode {} avec la commande
\texttt{source()}.
  En début de session seulement, donnez
  l'emplacement complet de la fonction \texttt{GS\_MotherCode.r} (donc
  remplacez ... par l'emplacement du \texttt{dossier.R}).


\begin{Schunk}
\begin{Sinput}
> source(".../GS_KNS_MotherCode.r")
\end{Sinput}
\end{Schunk}

\emph{Note}: La première fois que vous lancerez le code, il vous
faudra installer certaines librairies R (vous aurez comme message d'erreur:
\texttt{there is no package called ...}). \\
\\

\noindent {\color{red}Important!!} Vous devez charger le fichier \mcode {} au moins
une fois en début de session, car ce fichier installe toutes les fonctions nécéssaires à la production des
sorties et défini plusieurs
variables utilisées dans l'analyse.

\subsection*{Rajouter un filtre sur les espèces}
Si vous désirez inclure ou exclure de l'analyse certains groupes
taxonomiques, mettez l'option \texttt{filtre.sur.especes =
  TRUE}. Lorsque \mcode {} est lancé, R vous
demandera automatiquement de spécifier les valeurs du filtre sur
espèces (voir exemple ci-bas). Pour changer ces valeurs manuellement,
vous pouvez également lancez
vous même la fonction:

\begin{Schunk}
\begin{Sinput}
> filtre.especes("oui")
\end{Sinput}
\end{Schunk}

\texttt{\color{MidnightBlue} Inclure ou exclure? } \\
\indent \indent \texttt{Inclure } \\
\\
\indent \texttt{\color{MidnightBlue} Unité taxomique? (Groupe/Sous-Groupe/Famille/Genre/Espece)} \\
\indent \indent \texttt{Grp2}\\
\\
\indent \texttt{\color{MidnightBlue} Nom?} \\
\indent \indent \texttt{Crustaces, Mollusques, Echinodermes}\\
\\

\noindent Attention: lorsque \texttt{filtre.sur.especes =
  FALSE}, lancer \texttt{GS\_KNS\_MotherCode.r} ôtera automatiquement
le filtre enregistré au préalable. Pour ôter \emph{temporairement} ce comportement,
vous pouvez rajoutez un \# devant la ligne qui exécute cette fonction,
en prenant bien soin de l'ôter avant de quitter R pour qu'elle soit
active lors de la prochaine session. \\
\\
\texttt{\#if(!filtre.sur.especes)\{filtre.especes()\}else\{filtre.especes("oui")\}}

Pour voir la valeur des filtres présentement enregistrée, tapez
\texttt{voir.filtre()} dans la console R.

\subsection*{Modifier les requêtes lancées par défaut}
Le fichier \texttt{GS\_CodesInvertebres\_Launch.r} doit être édité
pour modifier les requêtes lancées par défaut.


\section{Description des fonctions pour invertébrés, poissons et LIT}
\subsection{Invertébrés}

\begin{itemize}

  \item[] \hypertarget{i1}{\bigtt{inv.biodiv(AS="pas de filtre",qunit="St",wC="all",save=FALSE)}:} \\
    \emph{Inv\_IndexBiodivPar + qunit + filtre station + filtre taxo} \\
  \textbf{Tableau} des indices de biodiversité (moyenne et écart type)
  et richesse spécifique (via \bigtt{inv.RichSpecifique()}) par station (valeur par défaut, peut être
  changée en modifiant l'argument \bigtt{qunit}). Le tableau produit
  est utilisé par d'autres fonctions, entre autres \texttt{inv.biodiv.geom()}.

  \item[] \hypertarget{i2}{\bigtt{inv.biodiv.geom(AS="A",save=FALSE)}:}\\
    \emph{Inv\_IndexBiodiv + géomorphologie/impact + filtre station + filtre taxo}\\
    \textbf{Tableau} résumé des moyennes (et ET) sur station par
    géomorphologie, et par géomorphologie/type d'impact.

  \item[] \hypertarget{i3}{\bigtt{inv.sprich.tbl(AS="A",grtax="Groupe",save=FALSE)}:}
    \emph{Inv\_NumEspeceParGeomorph + groupe taxo + filtre station +
      filtre taxo} \\
    \textbf{Tableau} du nombre d'espèces pour chacun des membres de
    l'aggrégation taxonomique spécifiée (\texttt{grtax}), proportion
    des espèces contenues dans chaque membre, et nombre total
    d'espèces observées par campagne/géomorphologie.

  \item[] \hypertarget{i4}{\bigtt{sprich.by.aggrtaxo(AS="A", grtax="Groupe", save=FALSE)}:}\\
    \emph{Inv\_NumEspeceParAggrTaxon + groupe taxo + filtre station + filtre taxo} \\
    \textbf{Tableau} de la richesse spécifique (nombre d'espèces) pour
 chaque membre du l'aggrégation taxonomique choisie
    (\texttt{grtax}). Par exemple si \texttt{grtax = "S\_Groupe"}, le
    tableau contiendra pour chaque station le nombre d'espèce pour
    chacun des sous-groupes taxonomiques observés sur les stations
    contenues dans le filtre.

  \item[]  \hypertarget{i5}{\bigtt{inv.RichSpecifique(AS="A", aj.impact=FALSE, aggr="St")}:}\\
    \emph{Non sauvegardé individuellement}
    \textbf{Tableau} interne non sauvegardé mais utilisé en
    information complémentaire pour d'autres tableaux
    (\texttt{inv.biodiv()}, \texttt{inv.biodiv.geom()}). Richesse
    spécifique par niveau taxonomique (nombre de genres, familles, sous-groupes,
    groupes) par station ou transect, avec option d'ajouter la moyenne
    (ET) de la RS par géomorphologie (\texttt{aggr="geom"}) et zone
    d'impact (\texttt{aj.impact = TRUE}).

   \item[] \hypertarget{i6}{\bigtt{inv.dens.tbl(AS="A", smpl.unit="St", grtax="G\_Sp",
      wZeroAll=FALSE, save=FALSE)}:}\\
\emph{Inv\_DensitePar\_ + smpl.unit + filtre station + groupe taxo + filtre taxo}\\
\textbf{Tableau} des densités moyennes (et SD) par station (ou
transect, si \texttt{smpl.unit = 'T'}) pour une
une unité taxonomique donnée (\texttt{grtax}), avec l'option de
rajouter dans le tableau final les densités nulles sur les
stations/campagne non-occupées (\texttt{wZeroAll}).

\item[] \hypertarget{i7}{\bigtt{inv.dens.geom(AS="A", aj.impact=FALSE,
      spttcampagnes=FALSE, save=FALSE)}:}\\
  \emph{Inv\_DensityTS\_Geomorpho + (Impact) + unité taxonomique +
    filtre stations + filtre taxonomique + type de top10}\\
  \textbf{Tableaux} des densités moyennes (et SD) par espèce, par
  sous-groupe et par groupe, pour chaque géomorphologie (et zone
  d'impact si \texttt{aj.impact = TRUE}). Les 10 espèces les plus
  abondantes durant la première, la
  dernière, et toutes les années de la série temporelle sont
  sélectionnées. Cette sélection est faite par groupe et par-sous
  groupe.

\item[] \hypertarget{i8}{\bigtt{inv.graph.TS(AS="A", wtype="allsp",
      wff="Groupe", top10year="", save=TRUE)}:}\\
  \emph{InvDensMoyTS\_ + filtre stations + type calcul top10 + année
    top 10 + niveau taxonomique + géomorpho} \\
  \textbf{Graphiques} des séries temporelles de densité (et SD) par
  unité taxonomique (\texttt{wff}) (un graphique par géomorphologie),
  avec option d'utiliser toutes les espèces (\texttt{wtype = 'allsp'})
  ou les dix espèces les plus abondantes en 2006 ou sur toute la série
  temporelle (\texttt{wtype = 'top10', top10year = 2006} ou
  \texttt{wtype = 'top10', top10year = 'all'}).


\item[] \hypertarget{i9}{\bigtt{inv.graph.TS.top10(AS="A",
          wff="Groupe", top10year="all", save=TRUE)}:}\\
        \emph{InvDensMoyTS\_ + filtre stations + top10 + top10year +
          niveau taxo + géomorpho + nom membre niveau taxo} \\

\textbf{Graphiques} des séries temporelles de densité (et SD) des 10
espèces les plus abondantes pour chaque membre de l'unité taxonomique
(\texttt{wff}) (un graphique par géomorphologie et
  membre de l'unité taxonomique). L'option \texttt{top10year} contrôle
  la selection des dix espèces les plus abondantes (en 20XX,
  \texttt{top10year = 20XX}) ou sur toute la série temporelle
  (\texttt{top10year = all}).

\end{itemize}

\subsection{Poissons}
\begin{itemize}
\item[] \hypertarget{p1}{\bigtt{poissons.tableau.brut(save=FALSE)}:}\\
\emph{GS\_Poissons\_TableauDonneesBrutes\_ + date + .csv}\\
Reformattage du \textbf{tableau} de données brutes initiales en incluant les
valeurs de densité et de biomasse pour chaque observation.


\item[] \hypertarget{p2}{\bigtt{BioDens.sp.poissons()}:} \\
Tableau préparatoire (non sauvegardé) calculant les densités et
biomasses par campagne, transect et espèce en se basant sur l'objet
\texttt{ds.calc} produit par la fonction
\texttt{poissons.tableau.brut()}. Le tableau produit par cette
fonction (nommé \texttt{BDtable} dans l'environnement R) est utilisé
par les fonctions \texttt{TS1.poissons()} et \texttt{TS2.poissons()}.

\item[]\hypertarget{p3}{\bigtt{poissons.ts1(AS="A",save=FALSE)}:}\\
\emph{GS\_Poissons\_TS1\_ + filtre + date + .csv}\\
 \textbf{Tableau} synthèse contenant la moyenne et l'écart type pour la
 densité, la biomasse, la richesse spécifique et la taille moyenne,
 calculés pour les catégories suivantes: toutes espèces confondues,
 commerciales, carnivores, herbivores, piscivores, planctonophages,
 sédentaires, territoriales, mobiles, très mobiles et ciblées (ou non)
 par la pêche en Nouvelle-Calédonie.

\item[]\hypertarget{p4}{\bigtt{poissons.ts2(AS="A",save=FALSE)}:}\\
  \emph{GS\_Poissons\_TS2\_ + filtre + date + .csv}\ \\
\texttt{Tableau} synthèse contenant pour chaque espèce la densité
moyenne sur toutes les années, pour chaque année, sur toutes les
stations et pour chaque station.

\item[]\hypertarget{p5}{\bigtt{poissons.p3(quel.graph="all",save=FALSE)}:}\\
 \emph{GS\_PoissonsST\_ [valeur représentée].[catégorie de poissons] +
   date + .pdf}\\
 Cette fonction produit des \textbf{graphiques} représentant des séries
 temporelles de densité, biomasse ou richesse spécifique pour les
 catégories de poissons définies dans l'objet
 \texttt{graph.key\$df.id}. Par défaut le filtre annuel est
 appliqué. Cette fonction est seulement executée dans le code
 \texttt{GS\_CodePoissons\_Launch.r} si la valeur de l'objet
 \texttt{graph.poissons} est changée à \texttt{TRUE}.

  \end{itemize}

\subsection{LIT}

\begin{itemize}
\item[] \hypertarget{l1}{\bigtt{LIT.tableau.brut(save=FALSE, AS="pas de filtre")}:}\\
  \emph{GS\_LIT\_TableauBrut\_ + filtre + date + .csv}\\
  \textbf{Tableau} formatté des données
de couverture brutes (une rangée par campagne/transect) -- ce
tableau est sauvegardé dans l'environnement R sous
\texttt{LITbrut} et est utilisé par les autres fonctions LIT.

\item[] \hypertarget{l2}{\bigtt{LIT.resume(yy=2011, ff="Coraux\_Gen", AS="A",
    save=FALSE)}}:\\
  \emph{GS\_LIT\_Tableau1A\_ + code catégorie de coraux + filtre + year + date + .csv}\\
  \textbf{Tableau} des couvertures moyennes (et SE) pour
  chaque géomorphologie selon l'année et la catégorie de substrat
  spécifiées (par défaut, 2011 et types de substrat généraux).

\item[] \hypertarget{l3}{\bigtt{LIT.ts1(AS="A")}}:\\
  Tableau 1: \emph{GS\_LIT\_SerieTempTable\_ + filtre + \_GeoMorphImpact\_ + date + .csv}\\
  Tableau 2: \emph{GS\_LIT\_SerieTempTable\_ + filtre + \_GeoMorphImpact\_DiffCouv\_ + date + .csv}\\
  Figures: \emph{GS\_LIT\_SerieTempTable\_ + filtre + \_GeoMorphImpact\_ + type de coraux + géomorphologie + date + .pdf}\\
  \textbf{Deux tableaux} sont produits. Le premier contient la
moyenne (et SE) des couvertures pour chaque campagne, géomorphologie et
zone d'impact, le deuxième contient la différence entre les
couvertures par géomorphologie des zones impactées et
non-impactées. Des \textbf{graphiques} sont également produits et
comparent pour chaque géomorphologie et catégorie de substrat les
tendances de couvertures moyennes par campagne entre les zones
impactées et non-impactées.

\item[] \hypertarget{l4}{\bigtt{LIT.ts2(AS="A")}}:\\
  Tableau: \emph{GS\_LIT\_SerieTempTable\_ + filtre + \_GeoMorphImpact\_bySt\_ + date + .csv}\\
  Figure: \emph{GS\_LIT\_SerieTemp\_ + filtre +
    \_GeoMorphImpact\_bySt\_ + type de coraux + géomorphogie +
    catégorie d'impact + date + .pdf}\\
  \textbf{Tableau} des couvertures moyennes (et SE) pour
chaque type de substrat par station et campagne. Cette fonction
produit aussi des \textbf{graphiques} comparant la couverture moyenne de chaque
type de substrat entre les stations de la même géomorphologie et zone
d'impact.

\item[] \hypertarget{l5}{\bigtt{LIT.bp1(yy=2011, ff2="Coraux\_Gen", AS="A")}}:\\
  Nom de fichier: \emph{GS\_LIT\_Hist1\_ + year + code catégorie de coraux
    + .pdf}\\
  \textbf{Graphiques} en barre comparant la
  couverture moyenne (et SE) à l'intérieur d'une catégorie de substrat
  et pour l'année spécifiée.

\end{itemize}

Note: les catégories de substrats sont définies dans l'objet
\texttt{coraux.fig}, créé par la fonction \texttt{prep.analyse()}.

\section{Concepts utiles sous R}
\subsection * {Fonctions}
Le terme \emph{fonction} sous R dénomme un objet contenant une ou plusieurs
lignes de codes qui performent une tâche spécifiée par le programmeur
(sous Excel on appellerait ça une macro). Une fonction
est très utile lorsqu'on veut répéter les mêmes lignes de code
en ne changeant qu'un seul aspect à chaque fois, ou lorsqu'on veut pouvoir
lancer une analyse complexe en ne tapant qu'une seule commande. Par exemple si je veux
faire le même graphique pour les Crustacés, Mollusques et
Échinodermes, je pourrais taper les même ligne de codes 3 fois en
changeant les termes `Crustaces', Mollusques', etc., ou bien je pourrais
écrire une fonction qui prend en \emph{argument} le groupe (taxonomique).
Par exemple:
\begin{Schunk}
\begin{Sinput}
> fonction.exemple <- function(x) x * 2
> fonction.exemple(4)
\end{Sinput}
\begin{Soutput}
[1] 8
\end{Soutput}
\end{Schunk}
\texttt{function()} est une fonction de base R utilisée pour
indiquer qu'on défini un objet de type ``function", qu'il s'appelle ``fonction.exemple",
qu'il prend en argument ``x" et qu'il multiplie cet argument par la
valeur 2.

\begin{Schunk}
\begin{Sinput}
> fonction.exemple()
\end{Sinput}
\end{Schunk}
\begin{Schunk}
\begin{Soutput}
Error in x * 2 : 'x' is missing
\end{Soutput}
\end{Schunk}
Ici ``x" n'a pas de valeur par défaut, donc R retourne un message
d'erreur. Lorsqu'on définit une valeur par défaut pour \texttt{x}
(e.g. \texttt{x=3}, voir ci-bas),
lancer \texttt{fonction.exemple()} ne retourne plus de message
d'erreur, mais bien la valeur $6$.
\begin{Schunk}
\begin{Sinput}
> fonction.exemple <- function(x=3) x * 2
> fonction.exemple()
\end{Sinput}
\begin{Soutput}
[1] 6
\end{Soutput}
\end{Schunk}

Pour voir les valeurs par défaut définies dans les fonctions,
vérifiez si les arguments sont suivis par un \texttt{$=$}, ou pas. Si vous lancez une
fonction et ne définissez pas un des arguments, R prendra la valeur
définie par défaut ou donnera un message d'erreur (si absente). A
faire attention, donc: lorsque vous lancez des fonctions manuellement
il faut s'assurer que les valeurs définies correspondent bien à ce que vous
voulez produire.


\begin{comment}
\section{Visualiser un tableau sous R}

De plus, pour visualiser un des tableaux produits directement sous R,
il suffit de lancer la commande manuellement dans la console comme
suit:

...

Pour voir les premières (ou dernières) 6 lignes du tableau, utiliser
la commande \texttt{head} (ou \texttt{tail}):

... et pour voir le tableau au complet vous tapez le nom de l'objet
dans la console.


Pour sélectionnez seulement les lignes où la station à la valeur ...

... et si vous vouliez seulement voir les colonnes ...

Les opérateurs logiques sous R sont:
== est égal à
!= n'est pas égal à
x \& y les deux conditions doivent être respectées
x | y l'une ou l'autre des conditions doivent être respectées
\end{comment}

\end{document}
