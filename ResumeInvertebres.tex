\documentclass{article}
\usepackage[utf8]{inputenc}
\usepackage{amsmath}
\usepackage{fullpage}
\usepackage{pdflscape}
\usepackage{verbatim}
\title{Programmation sous R de graphiques et tableaux de routine
  pour Ginger/SOPRONER}
\author{Laura Tremblay-Boyer}

\begin{document}

%\maketitle

\noindent {\Huge Invertébrés}

\section{Métriques de biodiversité}
\subsection{Richesse spécifique calculée à l'unité de la
  \emph{station}}
Nombre d'espèces, genres, familles, sous-groupes ou groupes  \emph{uniques} 
observés sur une même station, tous transects confondus. Les espèces
identifiées au niveau du genre (\emph{X. sp}) sont traitées comme une
espèce unique.

\subsection{Indices de Shannon ($H'$), Piélou ($J$) et Margalef
  ($d$)}

{\large Ces indices sont calculés à l'unité du transect:}

\subsubsection{Index de Shannon \emph{H}}

$ \displaystyle H' = \sum_i \left( \frac{N_i}{N} \right)
\ln \left( \frac{N_i}{N} \right) $ \\
\\
N est le nombre d'individus d'une espèce observés sur le transect (et
non la station), donc les observations nulles ($N_i=0$) sont
ignorées vu que $\ln(0)=-\infty$.\\


\subsubsection{Index d'équitabilté $J$}
$\displaystyle J=\frac{H'}{\ln S}$ \\
\\
$S$ est la richesse spécifique observée sur un transect. Lorsque
$S=1$, l'indice $J$ est égal à $\infty$. Les transects concernés sont
ignorés dans le calcul du $J$ moyen par station. 

\subsubsection{Index de Margalef $d$}
$\displaystyle d = \frac{S-1}{\ln N} $ \\
\\
Cet index n'existe que lorsque $N>1$, et les transects
ne satisfaisant pas cette condition sont ignorés lors du calcul de $d$
sur la station. 

\subsection{Assemblage par facteur}

Moyenne et écart-type par station: \\
$H', J, d$ pour une station sont la moyenne des valeurs correspondantes
sur les transects de la station. Dans les cas exceptionnels où certains transects sont vides
($N=0$), ces transects sont ignorés dans le calcul. 
\\
Moyenne et écart-type par impact et/ou géomorphologie: \\
$H', J, d$ par facteur groupant est calculé comme la moyenne des $H',
J, d$ moyens des stations du groupe. 

\subsection{Nombre d'espèce par géomorphologie ventilée par groupe}
Nombre d'espèces uniques observées sur une géomorphologie, toutes
stations confondues. 

% Insérer tableau: sorties demandées, méthode utilisée, nom des
% fonctions R utilisées, nom de
% fichier, vérification faite + date


\newpage

\section{Mesures de densité}
\subsection{Densité d'une espèce sur un transect $t$: }
$\displaystyle Dt = \sum_i N_i / (Ltrans_t * 50)$ \\
n est le nombre d'observations indépendantes de l'espèce sur le
transect\\
$Ltrans$ est la largeur du transect, la longueur est 50 mètres par
défaut. \\

\noindent Fonction: \texttt{inv.dens.sp( )} définie sous \texttt{GS\_CodesInvertebres\_Densite.r}

\subsection{Densité d'une espèce sur une station $s$:}

$\displaystyle D_s = 1/NT * \sum_t Dt $ \\
$D_t$ est la densité de l'espèce sur le transect t, $ t \in \{ A, B, C
\}$; 
$NT$ est le nombre de transect par station, pour les invertébrés NT =3. 
Lorsque l'espèce est observée sur quelques transects seulement sa
densité sur les autres transects est de zéro. 

\subsection{Densité moyenne d'une espèce sur une géomorphologie $geo$:}
$\displaystyle D_{e, geo} = 1/NSgeo * \sum_s D_s$ \\
$NSgeo$ est le nombre de stations sur la géomorphologie $g$; la densité de
l'espèce sur une station est zéro lorsque non-observée

\subsection{Densité moyenne d'un groupe $gr$ sur une
  géomorphologie $geo$:}
$\displaystyle D_{gr, geo} = 1/NE_{gr,geo} *\sum_e D_{e, geo}$ \\
$ e {} \in $ espèces du groupe \emph{gr} observées sur la géomorphologie; 

\subsection{Identification des 10 espèces les plus abondantes sur
  un groupement spatial}

\subsubsection{En utilisant la première ou la dernière campagne
  disponible sur la série temporelle}

Permet d'identifier les espèces dont l'abondance aurait changé
  de manière importante sur la période de l'échantillonage \\

$ \displaystyle AbMoy_{espece, grspatial, campagne} = \frac{AbTotale_{espece, grspatial,
    campagne}}{Nsite_{grspatial, campagne}}$ \\
\\
$Nsite_{grspatial}$ est le nombre de sites échantillonés sur le
groupement spatial $grspatial$ (géomorphologie $\pm$ impact),
\emph{campagne} est défini comme la première ou la dernière campagne
disponible dans la série de données (e.g. 2006 et 2011)

\subsubsection{En se basant sur la série temporelle complète}
Permet de suivre les espèces dominantes sur toute la période de
l'échantillonage.  Pour chaque espèce sur un groupement spatial \\
\\
$\displaystyle AbMoy_{espece,grspatial} = \frac{\sum_i AbMoy_{campagne_i} }{Nsite_{grspatial}}$ \\
$Nsite_{grspatial}$ est le nombre de sites échantillonés sur le
groupement spatial $grspatial$ (géomorphologie $\pm$ impact)

%%%%%%%%%%%%%%%%%%%%%%%%%%%%%%%%%%%%%%%%%%
%%%%%%%%%%%%%%%%%%%%%%%%%%%%%%%%%%%%%%%%%%
%%%%%%%%%%%%%%%%%%%%%%%%%%%%%%%%%%%%%%%%%%

\newpage
\begin{landscape}
\thispagestyle{empty}

\begin{table}
\caption{\label{allo} Description des fonctions R utilisées pour produire les
  sorties pour les invertébrés (l'unité temporelle est celle de la campagne)}
\begin{center}
\begin{tabular}{l l l}
\scriptsize
\\
\textbf{Fonction R} & \textbf{Fichier} & \textbf{Sorties} \\
\\
\hline
\hline
\\

\texttt{inv.biodiv()} & GS\_CodesInvertebres\_Biodiv.r & Indice de Shannon,
Index \emph{J} et Index \emph{d} (moyenne et ET) par station, \\
&& et richesse spécifique par unité taxonomique  \\
&&\\
\texttt{inv.biodiv.geom()} & GS\_CodesInvertebres\_Biodiv.r &
Indices de biodiversité (moyenne et ET) et richesse spécifique  \\
&& par unité taxonomique par géomorphologie (et impact)  \\
&&\\
\texttt{inv.sprich.tbl()} & GS\_CodesInvertebres\_Biodiv.r & Nombre
d'espèces par type d'un niveau taxonomique donné, par géomorphologie \\
&&\\
\texttt{inv.RichSpecifique()} & GS\_CodesInvertebres\_Biodiv.r & 
Richesse spécifique par unité taxonomique \\
&& (lancé par
\texttt{inv.biodiv()} et \texttt{inv.biodiv.geom()}) \\
&&\\
\hline
\hline
&&\\
\texttt{inv.dens.tbl.parT()} & GS\_CodesInvertebres\_Densite.r &
Densité par transect pour un niveau taxonomique donné, \\ 
&&
avec ou sans les densités nulles (option \texttt{wZeroAll}) \\ 
&&\\
\texttt{inv.dens.tbl()} & GS\_CodesInvertebres\_Densite.r & Densité
(moyenne et ET) par station pour un niveau taxonomique donné,\\
&&
avec ou sans les densités nulles (option \texttt{wZeroAll}) \\ 
&&\\
\texttt{inv.dens.geom()} & GS\_CodesInvertebres\_Densite.r & Densité
(moyenne et ET) par géomorphologie \\
&& (avec ou sans facteur \emph{Impact}, option \texttt{aj.impact})\\
&&\\
\texttt{inv.graph.TS()} & GS\_CodesInvertebres\_Densite.r & Séries temporelles
de la densité (moyenne et ET) des groupes ou sous-groupes, \\
&& par géomorphologie\\
&&\\
\texttt{inv.graph.TS()} & GS\_CodesInvertebres\_Densite.r & Séries temporelles
de la densité moyenne (et ET) des groupes ou sous-groupes, \\
&& par géomorphologie, en utilisant les espèces présente dans le top 10 seulement \\ 

\\
\hline
\\
\\

\end{tabular}
\end{center}
\label{table_inv}
\end{table}
\end{landscape}

%%%%%%%%%%%%%%%%%%%%%%%%%%%%%%%%%%%%%%%%%%
%%%%%%%%%%%%%%%%%%%%%%%%%%%%%%%%%%%%%%%%%%
%%%%%%%%%%%%%%%%%%%%%%%%%%%%%%%%%%%%%%%%%%

\begin{comment}
\section*{Fonctions clés}


\section*{Poissons}
Code sous: GSwork\_Start.r
A faire: nettoyer code poissons, incorporer fonction barres d'erreur,
changer les noms de colonne pour harmoniser avec sprich
\begin{enumerate}
\item Tableau brut
\item Tableau synth\`ese I
\item Tableau synth\`ese II
\item S\'eries temporelles I
\item S\'eries temporelles II
\end{enumerate}

\newpage


\section*{LIT}
\end{comment}

\end{document}