\documentclass{article}
\usepackage{fullpage}
\usepackage[utf8]{inputenc}
\usepackage{hyperref}
\usepackage{parskip}
\input{/Users/lauratb/Projects/misc-ressources/misc-latex-tools.tex}
\newcommand{\mcode} {\texttt{GS\_MotherCode\_TProj.r}{ }}
\hypersetup{colorlinks, 
linkcolor=PineGreen, 
filecolor=TealBlue, 
urlcolor=PineGreen, 
citecolor=NavyBlue}
\begin{document}

{\Large \textbf{Description des codes R développés pour les analyses
    des données d'échantillonage}}\\
Auteur: Laura Tremblay-Boyer, contact: lauratboyer@gmail.com\\
Nouméa le 25 février 2015\\

\section{À compléter}

vérifier format couleurs
relecture finale

\clearpage
\tableofcontents

\section{Survol: une session en exemple}
Pour une illustration rapide de l'utilisation des codes, vous
trouverez ci-dessous une série de \texttt{Commandes R} ainsi que les
\emph{explications} pour les arguments donnés. Ces fonctions et
arguments sont expliqués en détails dans les sections suivantes.\\

\texttt{> source(GS\_MotherCode\_TProj.r)} \\
\emph{On commence par charger les données et les fonctions sous R.}\\

\texttt{> selection.projet()}\\
\emph{Sélection du projet à analyser, KNS\_KONIAMBO par défaut, voir
  \texttt{nom.projets()} pour les autres options}\\

\texttt{> facteurs.tempo}\\
\texttt{[1] "Période.BACI" "Saison"}\\
\emph{Valeurs disponibles pour l'aggrégation temporelle, voir aussi
  \texttt{facteurs.spatio} et \texttt{facteurs.taxo}.}\\

%obj1 <- INV.dens.gnrl( par.transect=TRUE)

\texttt{> obj1 <- INV.dens.gnrl(fspat="Cote",agtaxo="Famille",
  filt.camp="A", save=TRUE)}\\
\emph{Densité moyenne (et ET) par Côte et Campagne (vu que l'argument
\texttt{ftemp} n'a pas été spécifié, on utilise l'aggrégation
temporelle par défaut, voir objet \hl{\texttt{ftempo.defaut}}). Le
tableau a été sauvegardé dans le dossier KNS\_KONIAMBO/Tableaux}\\

\texttt{> obj2 <- INV.biodiv.gnrl(ftemp=c("Saison","Campagne"),unit.base="St")}\\
\emph{Shannon, H et d (moyenne et ET), richesse taxonomique (totale, moyenne
et ET) par saison et station, en utilisant la station comme unité de
base (donc tous transects agrégés)  (l'argument
\texttt{fspat} n'a pas été spécifié, donc l'aggrégation spatiale est
sur les stations \hl{\texttt{fspat.defaut}}).
Mettre \texttt{ftemp=c(`Saison','Campagne')} fera l'aggrégation par
campagne et saison (mais vu que chaque campagne n'apparaît que dans
une saison ça ne fait que rajouter une colonne et les densités ne
changent pas), \texttt{fspat =
  c(`Geomorpho',`Cote',`St')} fera l'aggrégation spatiale par
station, côte et géomorphologie, \texttt{fspat =
  c(`Geomorpho',`N\_Impact')} fera l'aggrégation par géomorphologie et
zone d'impact, mais ici les densités seront différentes vu qu'il y a
plus qu'une zone d'impact par géomorphologie.}

\texttt{> obj3 <- POIS.dens.gnrl(fspat=`Geomorpho', agtaxo=`moblabel')}\\
\emph{Densité, biomasse, taille moyenne, richesse spécifique (et ET)
  par géomorphologie et catégorie de mobilité. Voir
  facteurs.taxo\$poissons pour les autres options d'aggrégation taxonomique.}

\texttt{> facteurs.taxo\$LIT}\\
\texttt{[1] "General"     "Forme"       "Acroporidae" "Sensibilite"
  "Famille" "Genre" }\\
\emph{Types de catégorie LIT disponible pour rafiner les analyses avec
l'argument \texttt{LIT.cat}}

\texttt{> obj4 <- LIT.couvrtr.gnrl(LIT.cat=`Acroporidae')}\\
\emph{Couverture (moyenne et ET) des Acroporidae/non-Acroporidae sur
  les transects LIT par station et campagne (\texttt{fspat} et
  \texttt{ftemp} n'ayant pas été spécifiés, les valeurs pas défaut
  sont utilisées)}\\

\texttt{> obj5 <- Quad.couvrtr.gnrl(fspat=`Geomorpho')}\\
\emph{Couverture (moyenne et ET) par catégorie LIT générale (valeur
  par défaut de \texttt{LIT.cat}) sur
  les quadrats par géomorphologie et campagne}\\

\section{Lancement initial}

La dernière version des codes peut être téléchargée sous le lien
suivant:\\

\url{https://github.com/lauratboyer/cw-ginger-soproner/archive/master.zip}\\

Repérez le dossier `Codes-tous-projets' et transférez-le à l'emplacement d'où vous
voulez lancer et sauvegarder vos analyses. Ouvrir
R et définir le répertoire courant à cet emplacement avec la commande
\texttt{setwd()}, ou en allant dans Fichier \ra Changer le répertoire
courant...\\

\texttt{setwd(`.../Codes-tous-projets') \# remplacer ``...''
par la trajectoire appropriée.}\\

Par exemple si
'Codes-tous-projets' est placé sur le bureau, le répertoire courant
devrait être défini comme suit:\\
\texttt{setwd(`C:/Users/gilbert/Desktop/Codes-tous-projets')}.

Pour confirmer que le changement a bien été fait, tapez
\texttt{getwd()} et confirmez que le retour correspond bien au dossier
voulu.

*** Si vous recevez un message d'erreur du genre:
\texttt{`Erreur dans getwd(`...'): impossible de changer le répertoire
  de travail'}, vous avez soit une erreur de frappe, soit
'Codes-tous-projets' n'est pas situé à l'emplacement spécifié.

Truc: si vous double-cliquez directement sur un des fichiers .r,
e.g. \mcode, le fichier s'affichera dans R
Studio et le répertoire de travail sera déjà mis à la valeur du
dossier contenant le code ouvert.

\section{Chargement des données}

Avant de commencer les analyses, vérifiez dans le fichier
\texttt{GS\_Mother-code\_TProj.r} la valeur des variables suivantes,
soit:
\begin{enumerate}
\item \hl{\texttt{facteurs.spatio}}: Cet objet contient les variables
  explicatives spatiales disponibles pour l'analyse (autre que
  \texttt{St}). Si vous rajoutez des variables dans le fichier
  \texttt{Facteurs\_spatiaux.csv}, vous devrez rajouter le nom de la
  nouvelle colonne dans cet objet pour que la variable soit prise en
  compte. En tout temps dans la session R vous pouvez consultez la
  valeur de cet objet pour voir les options qui peuvent être passées à
  l'argument \texttt{fspat} des fonctions d'analyses.
\item \hl{\texttt{facteurs.tempo}}: Comme pour les facteurs spatiaux,
  mais pour les variables explicatives temporelles (autre que
  'Campagne') disponibles dans le fichier
  \texttt{Facteurs\_temporels.csv}. Ces variables peuvent être données
  en argument sous \texttt{ftemp} dans les fonctions d'analyses.
\item \hl{\texttt{facteurs.taxo}}: Cet objet contient les valeurs
  disponibles pour les aggrégations taxonomiques (argument
  \texttt{ftaxo}) selon le groupe invertébrés, poissons ou LIT. Il
  existe purement pour un but
  informatif, donc vous ne devriez pas à avoir le changer (mais si
  vous le faites cela n'affectera pas les analyses).
\item \hl{\texttt{fspat.defaut}}, \hl{\texttt{ftempo.defaut}} et
  \hl{\texttt{agtaxo.defaut}}: Valeurs par défaut des facteurs
  spatiaux, temporels, et taxonomiques, utilisées lorsque les
  arguments \texttt{fspat}, \texttt{ftemp} ou \texttt{agtaxo} ne sont
  pas spécifiés en argument. Vous pouvez changez la valeur de ces
  objets selon vos préférences pour les analyses.
\item \hl{\texttt{filtre.annees}}: Un filtre sur les années pour
  l'analyse peut être spécifié ici mais je vous conseille de garder
  cette valeur à toutes les années (donc, \texttt{2006:2014}) quand vous
  initialisez les codes, et la changer ensuite manuellement dans la
  console R en tapant: \texttt{filtre.annees <- 2013}

\item \hl{\texttt{filtre.famille}}: Gardez cette valeur à
  \texttt{TRUE} pour appliquer automatiquement sur les invertébrés le
  filtre sur les familles spécifiées dans le ficher
  \texttt{Filtre-taxo\_Famille.csv}, sinon changez la valeur à
  \texttt{FALSE}.

\item \hl{\texttt{filtre.sur.especes}}: un filtre taxonomique peut
  être spécifié manuellement ici. Voir section \ref{ftaxo} pour les
  autres options.
\end{enumerate}

Une fois les valeurs des
variables changées (au besoin), sauvegardez les changements et
initialisez le chargement des fonctions R et des données avec la commande: \\
\texttt{source('GS\_Mother-code\_TProj.r')}\\

Ce code lance automatiquement la fonction \hllb{\texttt{import.tableaux()}},
qui importe les bases de données dans R, et la fonction
\hllb{\texttt{prep.analyse()}} qui nettoie les données pour l'analyse.

\subsubsection{Rajout de nouvelles données}
Si vous avez de nouvelles données à incorporer dans les bases R, sauvegarder les tableaux sous format .csv et mettez les nouvelles versions dans le dossier indiqué sous \texttt{dossier.DB} (et non \texttt{dossier.donnees}). Dans une nouvelle session R, allez dans \mcode, changez la valeur de la variable \hl{\texttt{refaire.tableaux}} à \texttt{TRUE} et lancer le script avec \texttt{source(...)}. Si le script tourne sans erreurs (vérifiez que de nouveaux objets ont été créés dans le dossier Tableaux-pour-analyses), re-changez la valeur de \hl{\texttt{refaire.tableaux}} à \texttt{FALSE} pour sauver du temps les prochaines fois lorsque vous lancez les scripts (les objets seront déja formattés, donc le chargement sera beaucoup plus rapide).
% si le chargement a déjà été fait vous pouvez sauvegarder une image...

\subsection{Filtre taxonomique} \label{ftaxo}

Le filtre sur espèces est défini par trois objets:

\begin{enumerate}
  \item \hl{\texttt{taxoF.incl}} spécifie si les noms donnés devraient être
    exclus (\texttt{`exclure'}) ou inclus (\texttt{`inclure'}) dans
    l'analyse.
  \item \hl{\texttt{taxoF.utaxo}} contient le niveau taxonomique à laquelle
    l'inclusion (ou l'exclusion) est faite, e.g. G\_Sp, Genre,
    Famille, S\_Groupe ou Groupe.
   \item \hl{\texttt{taxoF.nom}} contient la liste des noms des membres du
     niveau \texttt{taxoF.utaxo} sur laquelle l'action
     \texttt{taxoF.incl} sera portée.
\end{enumerate}

Une fois la session lancée, pour voir la valeur des filtres
présentement enregistrée, tapez \hllb{\texttt{voir.filtre.taxo()}} dans la
console R.

Vous avez plusieurs options pour changer la valeur du filtre:

\begin{itemize}
\item[--] Pour un lancement des codes complètement automatique, mettez dans
\mcode l'option \texttt{filtre.sur.especes =  TRUE} et modifiez
directement les valeurs des objets
\texttt{taxoF.incl}, \texttt{taxoF.utaxo} et \texttt{taxoF.nom}. *** Ces
valeurs peuvent aussi être modifiée manuellement dans la console R une
fois que \mcode{} est lancé.***
\item[--] \label{impFtaxo}Si vous avez une liste de noms à importer
d'un fichier .csv, utilisez la fonction \texttt{import.filtre.taxo()}
et spécifiez en argument \texttt{niveau=`...'} pour le niveau
taxonomique désiré et \texttt{action=exclure'} si cette action est
désirée (par défaut la fonction spécifie
\texttt{taxoF.incl='inclure'}. Pour voir le format approprié au
fichier .csv, référez-vous au fichier \texttt{Filtre-taxo\_Famille.csv}
\item[--] Finalement vous pouvez aussi utiliser
la fonction \texttt{def.filtre.especes()} pour définir intéractivement
sous R les valeurs du filtre, comme suit:


\texttt{$>$ def.filtre.especes.def("Inclure")}

\indent \texttt{\color{MidnightBlue} Unité taxonomique? (Groupe/Sous-Groupe/Famille/Genre/Espece)} \\
\indent \indent \texttt{Groupe}\\
\\
\indent \texttt{\color{MidnightBlue} Nom?} \\
\indent \indent \texttt{Crustaces, Mollusques, Echinodermes}\\
\end{itemize}

Pour éliminer le fitre sur espèces, faites
\texttt{def.filtre.especes()} dans la console R, sans arguments.

\subsection{Sélection du projet}

\ra \textbf{Cette étape est essentielle aux lancements des codes, sinon vous
aurez une erreur R.}\\

Une fois les données chargées et nettoyées, vous devez sélectionner le
projet à utiliser. Pour voir les noms de projet disponibles, faites:\\

\hllb{\texttt{nom.projets()}} \\

Vous pourrez sélectionner un de ces projets pour les analyses avec la
fonction \hllb{\texttt{selection.projet()}}, par exemple:\\

\texttt{selection.projet(`ADECAL\_TOUHO')}\\

Si vous ne donnez pas d'arguments à cette fonction, la valeur définie
par défaut est `KNS\_KONIAMBO'. Veillez à utiliser exactement le même
format défini sous \texttt{nom.projets()}.

En tout temps vous pouvez voir la valeur du projet en cours en tapant
la variable \hl{\texttt{projet}} dans la console R.

À noter que lorsqu'un projet est sélectionné, un dossier est
automatiquement créé (si non-existant) à l'intérieur du dossier
\texttt{Codes-tous-projets} pour sauvegarder les sorties tableaux ou
graphiques. Les bases de données sont filtrées pour ne conserver que
les tableaux pertinents au projet en cours, et mises dans les tableaux
\hl{\texttt{dbio}} pour les invertébrés, \hl{\texttt{dpoissons}} pour les
poissons, \hl{\texttt{dLIT}} pour les LIT et \hl{\texttt{dQuad}} pour les quadrats.

\section{Début des analyses}

Les fonctions suivantes font le calcul de la densité/couverture
moyenne et de l'écart type sur les aggrégations spatiales, temporelles
et taxonomiques définies en argument:
\begin{enumerate}
\item hop
\item \hllb{\texttt{INV.dens.gnrl()}}: fspat, ftemp, ftaxo
\item \hllb{\texttt{Pois.dens.gnrl()}}: fspat, ftemp, ftaxo
\item \hllb{\texttt{LIT.couvrt.gnrl()}}: fspat, ftemp, LIT.cat
\item \hllb{\texttt{Quad.couvrt.gnrl()}}: fspat, ftemp, LIT.cat
\end{enumerate}

Pour toutes les fonctions vous pouvez spécifiez `A' ou `S' à
l'argument \texttt{filt.camp} pour ne conservez que les campagnes
annuelles ou semestrielles, respectivement.

Pour voir les arguments disponibles pour n'importe quelle fonction (et leurs valeurs
par défaut), utilisez la fonction \hllb{\texttt{formals()}}, e.g:

\texttt{formals(INV.dens.gnrl)}

%Pour lancer une fonction de calcul, utilisez ... Pour utiliser les
%résultats dans R assignez la fonction à un objet.

Pour sauvegarder les calculs dans un tableau et les
visualiser/utiliser sous R, assignez la fonction à
un objet. Pour sauvegarder le tableau, spécifiez l'argument
\texttt{'save=TRUE'}. Par exemple: \\

\texttt{obj1 $<-$ INV.dens.gnrl() \# tableau contenu dans `obj1'}\\
\texttt{INV.dens.gnrl(save=TRUE) \# tableau sauvegardé mais non-visible dans R}\\
\texttt{obj1 $<-$ INV.dens.gnrl(save=TRUE) \# tableau contenu dans `obj1'}\\

\section{Invertébrés}

Fichier: \texttt{GS\_CodesInvertebres\_TProj.r}\\
Tableau formatté: \hl{\texttt{dbio}}\\
Fonctions:\\
\indent (1) \texttt{INV.dens.gnrl()}:  mesures de densité\\
\indent (2) \texttt{INV.biodiv.gnrl()}:  métriques de biodiversité\\

Les arguments \hl{\texttt{fspat}} et \hl{\texttt{ftemp}} définisent les facteurs
spatiaux et temporels d'aggrégation, respectivement, alors que
\hl{\texttt{agtaxo}} défini
le niveau taxonomique sur lequel les densités sont calculées. Les
valeurs par défaut pour ces arguments sont
`St', `Campagne' et `Groupe' (modifiables dans
\texttt{GS\_mother-code\_TProj.r}). Donc,
lancer \texttt{INV.dens.gnrl()} sans arguments spécifiques produira une moyenne (ET)
de la densité par groupe par station par campagne, alors que
\texttt{INV.dens.gnrl(fspat='Geomorpho', ftemp='Saison', agtaxo='Famille')}
produira ces mêmes statistiques mais par famille, géomophologie et
saison.

Pour avoir les valeurs par transect, donc sans aggrégation, utilisez
l'argument \hl{\texttt{`par.transect = TRUE'}}.

Abondances nulles: \hl{\texttt{wZeroT = TRUE}} (valeur par défaut) conserve
les abondances nulles sur tous les transects et campagnes où une
espèce n'a pas été observée. \hl{\texttt{wZeroSt = TRUE}} (mis à
\texttt{FALSE} par défaut) garde les abondances nulles sur les
stations même lorsque \texttt{wZeroT = FALSE}. Ça peut être utile si vous
voulez calculez les densités moyennes observées en excluant les
abondances nulles, mais quand même voir les stations où l'espèce n'a
été observée sur aucun des transects. Changer la valeur de
\texttt{wZeroSt} n'affectera le résultat seulement lorsque
\texttt{wZeroT = FALSE}.

Les arguments sont les mêmes pour \texttt{INV.biodiv.gnrl()}, à part qu'il n'y
pas d'aggrégation taxonomique \texttt{ftaxo}. Les métriques sont calculées à
l'échelle de l'espèce en utilisant soit le transect (valeur par
défaut, \hl{\texttt{unit.base = `T'}}) ou la station
(\hl{\texttt{unit.base = `St'}}) comme unité de
base pour le calcul. Les richesses spécifiques sont calculées à toutes
les échelles taxonomiques.

\section{Poissons}
Fichier: \texttt{GS\_CodesPoissons\_TProj.r}\\
Tableau formatté: \hl{\texttt{dpoissons}}\\
Fonctions:\\
\texttt{POIS.dens.gnrl()}: mesures de densité, biomasse, taille
moyenne et richesse spécifique\\

La fonction \texttt{POIS.dens.gnrl()} produit les analyses pour
les poissons. Les arguments principaux sont les mêmes que pour les
statistiques sur
les invertébrés, à part que l'argument \hl{\texttt{`agtaxo'}} peut-être utilisé
pour spécifier une échelle taxonomique ou une charactéristique
écologique, e.g. \texttt{agtaxo=`moblabel'} rendra les statisques par
type de mobilité comme spécifié dans le tableau Bioeco. Pour conserver
les abondances nulles sur les transects vous utilisez, comme pour les
invertébrés, \hl{\texttt{wZeroT = TRUE}} (valeur par
défaut). A noter que \texttt{wZeroSt} n'est pas disponible pour les
poissons vu qu'il n'y a qu'un transect par station dans la grande majorité
des cas.

\section{LIT et Quadrats}
Fichier: \texttt{GS\_CodesLIT\_TProj.r}\\
Tableaux formattés: \hl{\texttt{...}}\\
Fonctions:\\
\texttt{LIT.couvrt.gnrl()} et \texttt{Quad.couvrt.gnrl()}: couverture
moyenne par typologie LIT\\

Les statistiques LIT/Quadrat peuvent être obtenues avec les fonctions
\texttt{LIT.couvrt.gnrl()} et \texttt{Quad.couvrt.gnrl()}. L'argument
\hl{\texttt{LIT.cat}} spécifie
le type de catégorie LIT utilisé pour l'aggrégation (défini dans le
tableau `Type\_LIT.csv'). Les arguments \hl{\texttt{fspat}} et
\hl{\texttt{ftemp}} sont utilisés comme ci-dessus. Note: pour avoir les
données brutes par transect/quadrat, sans aggrégation, vous pouvez
utiliser les fonctions \texttt{LIT.tableau.brut()} et
\texttt{Quad.tableau.brut()}.

\section{Abondances nulles}
A partir du moment où une espèce est observée sur un transect pour un
projet, une abondance nulle est rajoutée sur tous les transects,
stations et campagnes où l'espèce est absente. 
% a verifier
%Ceci est valide pour 
%\hl{pro sur certains
%transects seulement d'une station/campagne, on rajoute une abondance
%nulle pour l'espèce pour les transects manquants de la campagne (donc,
%pas d'abondances nulles pour les campagnes précédentes si l'espèce
%était absente).}


\section{Création de tableaux formattés}


\section{Trucs pour trouver la source de l'erreur}

Avant tout travail de détective, commencez par vous assurez que vous
n'avez faites aucune erreur de frappe (!) et/ou spécification du
répertoire de travail.

S'il y a un bug lors du lancement des codes (initialement ou dans les
analyses subséquentes), commencez par identifier la fonction où le bug
apparaît. Typiquement ça sera indiqué dans le message d'erreur
(e.g. \texttt{Erreur dans la fonction LIT.couvrt.gnrl()}). Sinon
regardez bien le message pour identifier un des objets où l'erreur
prend place, e.g. \\

\texttt{Error in data.frame(t1[, colcl != "matrix"], t1.sub)
\hl{(from con\#756)}}\\

nous indique qu'il y a une commande, quelque part dans les codes, où
la ligne suivante est utilisée: \texttt{data.frame(t1[, colcl !=
  "matrix"], t1.sub)}, possiblement sur la ligne 756 d'un des fichiers
de code.

La fonction \hllb{\texttt{file.scan()}} permet de retrouver le fichier R où
une certaine ligne de code apparaît, comme suit: \\
\texttt{> file.scan("data.frame(t1"))}\\
\texttt{[1] "GS\_ExtractionDonnees\_TProj.r"}\\

... et effectivement si je vais vers la ligne 756 de ce fichier, je
retrouve cette commande, qui se trouve en fait dans une fonction
s'appelant \texttt{aggr.multi()}. (Notez que la fonction \texttt{file.scan()}
ne vient avec R par défaut mais est définie lors du lancement initial
des codes.)

Alternativement, si vous connaissez déjà le nom de la fonction où le
bug prend place, vous pouvez utilisez la fonction
\hllb{\texttt{debugonce(...)}}:\\
\texttt{> debugonce(prep.analyse)}\\

... qui vous permettra de lancer les commandes ligne par ligne (vous
devrez peser sur la touche retour pour que la ligne suivante soit
lancée dans la console), et donc identifier la ligne exacte où le bug
prend place.

Une fois la ligne identifiée, examinez les objets utilisés dans la
commande. En continuant avec l'exemple précédent, vous pourriez
faire:\\
\texttt{> debugonce(aggr.multi)}\\
et relancer la commande initiale qui avait produit le bug. Lorsque la
fonction arrivera à l'étape d'utilisation de \texttt{aggr.multi()}
elle devrait s'arrêter et vous faire passer par chaque ligne. Quand
vous reconnaissez la ligne où le bug s'est produit, avant de peser sur
la touche RETOUR, examinez en détail la commande...
\texttt{ t1.fn = data.frame(t1[,colcl!="matrix"], t1.sub)}
Ici on sélectionne les colonnes \texttt{colcl} dans le tableau
\texttt{t1}, et on le joint avec le tableau \texttt{t1.sub}. Vu que le
message d'erreur disait:\\
\texttt{arguments imply differing number of rows: 564, 0}\\
on peut se douter que \texttt{t1} et \texttt{t1.sub} n'ont pas le même
nombre de rangées... et effectivement lorsqu'on vérifie avec les
commandes \texttt{nrow(t1); nrow(t1.sub)} c'est bien le
cas. \texttt{t1.sub} est vide. L'étape suivante est de trouver où
\texttt{t1.sub} est crée, et de vérifier pourquoi il est vide... et
ainsi de suite. Tant que vous y allez étape par étape vous allez
trouver la source du bug, et deviendrez plus efficace avec la pratique
dans l'interprétation du message d'erreur. Ici, la commande initiale
ayant causée le bug était: \\
\texttt{> LIT.couvrt.gnrl(LIT.cat="hop")}\\
et éventuellement vous auriez trouvez que \texttt{t1.sub} est vide
parce qu'il n'y pas de catégorie s'appellant `hop' dans les tableaux LIT.

Finalement, il y a plusieurs fonctionalités pour `débugger' à explorer
dans RStudio.

\begin{comment}
\section{Informations supplémentaires}
\subsection{Fichiers de code R requis}

%GS\_MotherCode\_TProj.r
%GS\_MisesAJourDB\_TProj.r
%GS\_ExtractionDonnees\_TProj.r
%GS\_Selection-Donnees-Par-Projet.r
%GS\_CodesInvertebres\_TProj.r
%GS\_CodesPoissons\_TProj.r
%GS\_CodesLIT\TProj.r

\end{comment}

\end{document}